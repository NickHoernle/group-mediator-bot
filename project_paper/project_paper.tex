%%
%% Copyright 2007, 2008, 2009 Elsevier Ltd
%%
%% This file is part of the 'Elsarticle Bundle'.
%% ---------------------------------------------
%%
%% It may be distributed under the conditions of the LaTeX Project Public
%% License, either version 1.2 of this license or (at your option) any
%% later version.  The latest version of this license is in
%%    http://www.latex-project.org/lppl.txt
%% and version 1.2 or later is part of all distributions of LaTeX
%% version 1999/12/01 or later.
%%
%% The list of all files belonging to the 'Elsarticle Bundle' is
%% given in the file `manifest.txt'.
%%

%% Template article for Elsevier's document class `elsarticle'
%% with harvard style bibliographic references
%% SP 2008/03/01
%%
%%
%%
%% $Id: elsarticle-template-harv.tex 4 2009-10-24 08:22:58Z rishi $
%%
%%
\documentclass[final,authoryear,11pt,times]{elsarticle}

%% Use the option review to obtain double line spacing
%% \documentclass[authoryear,preprint,review,12pt]{elsarticle}

%% Use the options 1p,twocolumn; 3p; 3p,twocolumn; 5p; or 5p,twocolumn
%% for a journal layout:
%% \documentclass[final,authoryear,1p,times]{elsarticle}
%% \documentclass[final,authoryear,1p,times,twocolumn]{elsarticle}
%% \documentclass[final,authoryear,3p,times]{elsarticle}
%% \documentclass[final,authoryear,3p,times,twocolumn]{elsarticle}

%% \documentclass[final,authoryear,5p,times,twocolumn]{elsarticle}

%% if you use PostScript figures in your article
%% use the graphics package for simple commands
%% \usepackage{graphics}
%% or use the graphicx package for more complicated commands
%% \usepackage{graphicx}
%% or use the epsfig package if you prefer to use the old commands
%% \usepackage{epsfig}

%% The amssymb package provides various useful mathematical symbols
\usepackage{amssymb}

\usepackage[margin=1.25in]{geometry}
\usepackage{bbm}

\usepackage{setspace}
\onehalfspacing

%% The amsthm package provides extended theorem environments
%% \usepackage{amsthm}

%% The lineno packages adds line numbers. Start line numbering with
%% \begin{linenumbers}, end it with \end{linenumbers}. Or switch it on
%% for the whole article with \linenumbers after \end{frontmatter}.
%% \usepackage{lineno}

%% natbib.sty is loaded by default. However, natbib options can be
%% provided with \biboptions{...} command. Following options are
%% valid:

%%   round  -  round parentheses are used (default)
%%   square -  square brackets are used   [option]
%%   curly  -  curly braces are used      {option}
%%   angle  -  angle brackets are used    <option>
%%   semicolon  -  multiple citations separated by semi-colon (default)
%%   colon  - same as semicolon, an earlier confusion
%%   comma  -  separated by comma
%%   authoryear - selects author-year citations (default)
%%   numbers-  selects numerical citations
%%   super  -  numerical citations as superscripts
%%   sort   -  sorts multiple citations according to order in ref. list
%%   sort&compress   -  like sort, but also compresses numerical citations
%%   compress - compresses without sorting
%%   longnamesfirst  -  makes first citation full author list
%%
%% \biboptions{longnamesfirst,comma}

% \biboptions{}

\journal{cs280r - Final Project Report}

\begin{document}

\begin{frontmatter}

%% Title, authors and addresses

%% use the tnoteref command within \title for footnotes;
%% use the tnotetext command for the associated footnote;
%% use the fnref command within \author or \address for footnotes;
%% use the fntext command for the associated footnote;
%% use the corref command within \author for corresponding author footnotes;
%% use the cortext command for the associated footnote;
%% use the ead command for the email address,
%% and the form \ead[url] for the home page:
%%
%% \title{Title\tnoteref{label1}}
%% \tnotetext[label1]{}
%% \author{Name\corref{cor1}\fnref{label2}}
%% \ead{email address}
%% \ead[url]{home page}
%% \fntext[label2]{}
%% \cortext[cor1]{}
%% \address{Address\fnref{label3}}
%% \fntext[label3]{}

\title{CS280r Final Project Report \\ Project Name}

%% use optional labels to link authors explicitly to addresses:
%% \author[label1,label2]{<author name>}
%% \address[label1]{<address>}
%% \address[label2]{<address>}

\author{Anna Sophie Hilgard and Nicholas Hoernle}


\begin{abstract}
%% Text of abstract

\end{abstract}
\end{frontmatter}

% \linenumbers

%% main text
\section{Introduction}
\label{sec:introduction}

As has been shown in many studies of human interaction, communication is not free. In \citet{amir2015care}, the importance of efficient communication was stressed, as study participants reported that when provided with complete plans, they could not review the information in a timely manner. Similarly, we see in \citet{hahn2016knowledge} that crowdsourcing tasks consistently struggle with providing an amount of context that allows workers to complete an assignment while not spending the majority of their allotted task time getting up to speed. An added complication is that in many settings, the ideal contextual information to provide is subjective and multiple parties have competing interests in having their contributions addressed. While we could allow for a single contributor or an outside controller to make these subjective decisions, past experiences with content generators like Wikipedia with a strong hierarchical or dictatorial leadership \citet{benkler2015peer}  (in particular our in-class experience) have shown that the resulting content is often suboptimal from the viewpoint of the whole and heavily skewed to conform to the opinion(s) of the decision-maker(s). \citet{schwartz2015design} stresses that those situations in which group members have different information and the actions of individuals are interdependent are the most critical to be collectively assessed.

Under these conditions, we see a strong case for adopting social budgeting techniques to crowdsource contextual points. As shown in \citet{boutilier2015optimal}, if we assume adopt a utilitarian framework in which we hope to maximize the satisfaction of all group members, properly chosen voting rules can ensure that we minimize the maximum difference between the optimal possible satisfaction to all members and that selected by the voting rule in expectation (the regret), whereas it is clear that for a dictatorial selection this could be trivially equal to the worst case if the size of the alternative set is larger than two times the size of the set of options to be selected. In particular, we will seek to test the effectiveness of the subset selection algorithm generated by \citet{caragiannis2017subset}, which approaches the problem as a variation on the maximin rule. In particular, the authors show that it is possible to derive an explicit utility function which maximizes regret while maintaining consistency with the votes, leading to the following expression for maximum regret for a subset selection $T$:
$$
\textrm{max}_{S \in A_k} \sum_{i=1}^{n} \frac{\mathbbm{1}[S \succ_{\sigma_i}T]}{\sigma_i(S)}
$$
Where $S \succ_{\sigma_i}T$ indicates that there is no alternative in T preferred to every alternative in S given the utility function $\sigma_i$, and $\sigma_i(S)$ is the ordinal ranking of the best alternative in set $S$ in the ranking determined by the utility function $\sigma_i$.

Intuitively, any term in this maximization captures the lost satisfaction to the voters of not having the given set $S_i$ chosen rather than $T$, weighted by how much he or she liked his or her best option in $S_i$. This will lead to a greater penalization for sets $T$ that do not give many participants at least one of their top choices.

We seek the set $T$ that minimizes this quantity.
$$
\textrm{argmin}_{T \in A_k} \textrm{max}_{S \in A_k} \sum_{i=1}^{n} \frac{\mathbbm{1}[S \succ_{\sigma_i}T]}{\sigma_i(S)}
$$

Shah et al. show that this can be solved through an ILP with $n � m$ variables and $n � m^{2} $ + $n\choose m$ constraints, where $n$ is the number of voters and $m$ is the number of alternatives available.

For comparison, we also use plurality/knapsack voting, which has been used in real-world participatory budgeting programs (likely in part because of its computational simplicity and ease of understanding) \citet{goel2015knapsack} and is shown in Shah et al. to have empirical regret approaching that of the subset selection algorithm above for subset sizes greater than three, which will be the case in our experiment and should be generally true for problems of this nature.
%Definition 1 (Comparing Sets). Given a ranking ? ? L and an alternative a ? A, recall that ?(a) denotes the position of a in ?. More generally, for a set S ? A let ?(S) = mina?S ?(a). For sets S,T ? A, we say T ?? S if ?(T) < ?(S), i.e., if there exists an alternative in T that is preferred to every alternative in S in ?.

%The need for such efficient communication was evi- dent in our study: some of the providers we interviewed re- ported that when complete plans or notes are sent to them, they are unable to determine the information most important to consider, and they do not review the information in a timely manner as a result of this information overload. 

%Previous crowdsourc- ing approaches have trouble dealing with cross-topic consis- tency because reading even a single topic can take significant time, let alone reading and editing across all topics.

%These are often ones in which
%individuals must coordinate their actions because their parts of the organization are
%interdependent.

%Correctly specifying an agenda helps to keep meeting topics on track, on time and helps to improve the efficiency and effectiveness of the meeting \citep{schwartz2015design} \citep{lehmann2013sequential}. \citet{schwartz2015design} continues by stating the importance of including items in the agenda that reflect the needs of the individuals in the team. \citet{lehmann2013sequential} assert that unmanaged social interaction leads to poor decision making, ineffective communication and unnecessary conformity. In this light, we have revisited the Care Coordination work presented by \citet{amir2015care} and focus on designing a human-computer collaboration tool that is able to manage the creation of an agenda that is relevant for a diverse team of individuals. \citep{arber2008team} is in agreement with the thesis here that meetings significantly affect the performance of the medical team and on the care that the patient receives. The purpose of this system is to extend \citet{procaccia2016voting} `Voting Rules' framework and text mining techniques to create an initial set of structured agenda items that can then be agreed upon in advance. Furthermore, some of the specific obstacles to efficient teamwork demonstrated in the \citet{amir2015care} study shows that members of the team often find large meetings unproductive and not related to the work that they are doing with the patient. Having a structured agenda in advance will help the team members filter through the potentially time wasting and irrelevant content such that they can address the pertinent issues at hand. Following \citet{schwartz2015design} framework for creating an effective meeting agenda, we propose an agent that is able to source the relevant agenda items, action points and required duration from the team prior to the meeting occurring.
%
%\citep{fatima2004agenda} present a view of negotiations where agents all benefit by reaching an agreement but all have conflicting ideas over how certain tasks should be executed. In a similar way, agents in \citet{amir2015care} medical setting may have differing opinions on the best and most important treatment plans that require discussion, but reaching a feasible agreement in advance... 

\section{Body of the Paper}

\begin{itemize}
	\item {\bf  Experimental Design.} A description of the experiment that was run; enough detail should be provided   
that the reader could reasonably duplicate the experiment. Results should not be 
reported in this section.
	\item {\bf Results.} A report of the results of the experiments, and their significance.
\end{itemize}

\subsection{Citations}

Here are two examples of how to cite a paper properly:
\begin{itemize}
	\item \citet{bernstein2000complexity} shows that ... 
	\item Prior work has shown that ... \citep{bernstein2000complexity}.
\end{itemize}


%%  \citet{key}  ==>>  Jones et al. (1990)
%%  \citep{key}  ==>>  (Jones et al., 1990)


\section{Related Work}
Discussion of previous important, similar work in the area with comparison to the particular approach taken and results of the paper. Avoid simply providing a laundry list of other work that is somehow related to the subject of the paper. This section should contain brief, in depth discussions of the work most similar to your project, i.e., to research that takes an approach to the problem or produces results with which your project should be compared. As is always the case with written work, throughout the paper you should have citations to work that you draw on. For example, if you have adapted a system, include a citation to the system when you first mention it; if you are extending a formalization, include a citation to the original on first mention. If you are unclear about whether a simple citation suffices or an extended discussion is needed in the Related Work section, look at the papers read for class this semester for models. If you are still unsure, check with the teaching staff.

\section{Conclusion}
  Describes the insights that can be taken away from the work reported in the paper.

\section {Future work}
  Suggests extensions or challenges raised by the project.


%% The Appendices part is started with the command \appendix;
%% appendix sections are then done as normal sections
%% \appendix

%% \section{}
%% \label{}

%% References
%%
%% Following citation commands can be used in the body text:
%%
%%  \citet{key}  ==>>  Jones et al. (1990)
%%  \citep{key}  ==>>  (Jones et al., 1990)
%%
%% Multiple citations as normal:
%% \citep{key1,key2}         ==>> (Jones et al., 1990; Smith, 1989)
%%                            or  (Jones et al., 1990, 1991)
%%                            or  (Jones et al., 1990a,b)
%% \cite{key} is the equivalent of \citet{key} in author-year mode
%%
%% Full author lists may be forced with \citet* or \citep*, e.g.
%%   \citep*{key}            ==>> (Jones, Baker, and Williams, 1990)
%%
%% Optional notes as:
%%   \citep[chap. 2]{key}    ==>> (Jones et al., 1990, chap. 2)
%%   \citep[e.g.,][]{key}    ==>> (e.g., Jones et al., 1990)
%%   \citep[see][pg. 34]{key}==>> (see Jones et al., 1990, pg. 34)
%%  (Note: in standard LaTeX, only one note is allowed, after the ref.
%%   Here, one note is like the standard, two make pre- and post-notes.)
%%
%%   \citealt{key}          ==>> Jones et al. 1990
%%   \citealt*{key}         ==>> Jones, Baker, and Williams 1990
%%   \citealp{key}          ==>> Jones et al., 1990
%%   \citealp*{key}         ==>> Jones, Baker, and Williams, 1990
%%
%% Additional citation possibilities
%%   \citeauthor{key}       ==>> Jones et al.
%%   \citeauthor*{key}      ==>> Jones, Baker, and Williams
%%   \citeyear{key}         ==>> 1990
%%   \citeyearpar{key}      ==>> (1990)
%%   \citetext{priv. comm.} ==>> (priv. comm.)
%%   \citenum{key}          ==>> 11 [non-superscripted]
%% Note: full author lists depends on whether the bib style supports them;
%%       if not, the abbreviated list is printed even when full requested.
%%
%% For names like della Robbia at the start of a sentence, use
%%   \Citet{dRob98}         ==>> Della Robbia (1998)
%%   \Citep{dRob98}         ==>> (Della Robbia, 1998)
%%   \Citeauthor{dRob98}    ==>> Della Robbia


%% References with bibTeX database:

\bibliographystyle{elsarticle-num-names}
\bibliography{bibliography}

\end{document}

