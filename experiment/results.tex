\documentclass[10pt]{article}
%	options include 12pt or 11pt or 10pt
%	classes include article, report, book, letter, thesis
\usepackage[margin=0.3in]{geometry}

\begin{document}

\section{Question 1, Type: rate}
\textbf{[1, 2, 3, 7, 8]}
\begin{itemize}
\item Current admission guidelines for US colleges admissions offices result in the colleges failing to admit many top quality students.
\item Cumulative high school grade is widely used in the admissions process but grade point average (GPA) in senior years is more predictive of college success than high GPA in junior years.
\item Average ACT scores are widely used in college admissions but scores for only the Math and English sections are more predictive of student success in college.
\item There is currently an inherent disconnect between admissions offices and graduation rates (e.g. there is no direct feedback from graduation to the admissions decisions that were made four years prior).
\item Admissions officers should be more engaged in the long term goals of the university rather than optimizing for world rankings.
\end{itemize}

\section{Question 1, Type: sort}
\textbf{[1, 2, 3, 6, 7]}
\begin{itemize}
\item Current admission guidelines for US colleges admissions offices result in the colleges failing to admit many top quality students.
\item Cumulative high school grade is widely used in the admissions process but grade point average (GPA) in senior years is more predictive of college success than high GPA in junior years.
\item Average ACT scores are widely used in college admissions but scores for only the Math and English sections are more predictive of student success in college.
\item Improved admissions, resulting in improved graduation rates, will show in world rankings four years after initial implementation.
\item There is currently an inherent disconnect between admissions offices and graduation rates (e.g. there is no direct feedback from graduation to the admissions decisions that were made four years prior).
\end{itemize}

\section{Question1, Type: Plurality}
\textbf{[2, 5, 7, 1, 9]}
\begin{itemize}
\item Cumulative high school grade is widely used in the admissions process but grade point average (GPA) in senior years is more predictive of college success than high GPA in junior years.
\item Schools should be unconcerned about brief fluctuations in college ranking scores (caused by different admissions criteria), if it results in a stronger student body.
\item There is currently an inherent disconnect between admissions offices and graduation rates (e.g. there is no direct feedback from graduation to the admissions decisions that were made four years prior).
\item Current admission guidelines for US colleges admissions offices result in the colleges failing to admit many top quality students.
\item Currently colleges are not taking advantage of advanced statistical methods that can be used to predict student success.
\end{itemize}

\section{Question 2, Type: rate}
\textbf{[1, 4, 7, 8, 10]}
\begin{itemize}
\item Circumstance may be highly misleading in an interview scenario as an interviewee's demeanor may depend highly on external, hidden circumstances.
\item Interviews can actually be harmful to the hiring process, undercutting the impact of other, more valuable information about interviewees.
\item When interviewees respond randomly to interview questions, the interviewer has a strong belief that she `got to know' the interviewee (even though the responses have no bearing on the actual beliefs of the interviewee).
\item Interviewers naturally turn irrelevant information into a coherent narrative, biasing their conclusions.
\item Interviews should be used to test job related skills.
\end{itemize}

\section{Question 2, Type: sort}
\textbf{[2, 4, 6, 7, 10]}
\begin{itemize}
\item An interviewee's conduct may be interpreted differently by different interviewers.
\item Interviews can actually be harmful to the hiring process, undercutting the impact of other, more valuable information about interviewees.
\item When interviewees respond randomly to interview questions, the interviewer is unable to detect this trend.
\item When interviewees respond randomly to interview questions, the interviewer has a strong belief that she `got to know' the interviewee (even though the responses have no bearing on the actual beliefs of the interviewee).
\item Interviews should be used to test job related skills.
\end{itemize}


\section{Question2, Type: Plurality}
\textbf{[4, 3, 1, 2, 8]}
\begin{itemize}
\item Interviews can actually be harmful to the hiring process, undercutting the impact of other, more valuable information about interviewees.
\item Unstructured, `get-to-know' interviews are becoming popular in the workspace and in college admissions, yet these form a poor metric for predicting the future job performance of the interviewee.
\item Circumstance may be highly misleading in an interview scenario as an interviewee's demeanor may depend highly on external, hidden circumstances.
\item An interviewee's conduct may be interpreted differently by different interviewers.
\item Interviewers naturally turn irrelevant information into a coherent narrative, biasing their conclusions.
\end{itemize}

\section{Question 3, Type: rate}
\textbf{[3, 4, 5, 6, 8]}
\begin{itemize}
\item The fact that a pleasurable activity released dopamine is uninformative, as dopamine is released while playing video games, taking drugs and while partaking in any form of pleasurable activity.
\item The American Journal of Psychiatry has published a study showing that at most 1 percent of video game players might exhibit characteristics of an addiction.
\item The American Journal of Psychiatry has published a study showing that gambling is more addictive than video games.
\item The American Journal of Psychiatry has published a study showing that the mental and social heath of the purported video game addicts is no different from individuals who are not addicted to video games.
\item We and our children are `addicted' to new technologies because they improve our lives or are plainly enjoyable to use.
\end{itemize}

\section{Question 3, Type: sort}
\textbf{[1, 2, 3, 7, 10]}
\begin{itemize}
\item Video gaming is not damaging or disruptive to one's life and thus should not be compared to a drug.
\item Dopamine levels that are released while playing video games are vastly lower than those released while taking a drug such as methamphetamine.
\item The fact that a pleasurable activity released dopamine is uninformative, as dopamine is released while playing video games, taking drugs and while partaking in any form of pleasurable activity.
\item Treating the immoderate playing of video games as an addiction is pathologizing relatively normal behavior.
\item Using video gaming to relax does not constitute an addiction in much the same way as watching sports is not addictive.
\end{itemize}

\section{Question3, Type: Plurality}
\textbf{[9, 3, 4, 10, 6]}
\begin{itemize}
\item Evidence for addiction to video games is virtually nonexistent.
\item The fact that a pleasurable activity released dopamine is uninformative, as dopamine is released while playing video games, taking drugs and while partaking in any form of pleasurable activity.
\item The American Journal of Psychiatry has published a study showing that at most 1 percent of video game players might exhibit characteristics of an addiction.
\item Using video gaming to relax does not constitute an addiction in much the same way as watching sports is not addictive.
\item The American Journal of Psychiatry has published a study showing that the mental and social heath of the purported video game addicts is no different from individuals who are not addicted to video games.
\end{itemize}

\section{Qualitative Feedback}

\begin{itemize}
\item Ranking was quite challenging, especially given that several statements were highly similar or conceptually related. I think it would be helpful to have some very stupid arguments thrown in so that you can have greater variability in your measurement. Glad to see you taking an interest in social psychology, little sister!
\item ranking is difficult, scoring and/or selecting is easier
\item Ticking the 5 most relevant points to support my argument is the easiest form of feedback. Placing the different points in order is the most difficult and requires the most time as you have to read through each point multiple times in order to make comparisons and form the list in the order you want. Ranking the points on a scale of 1-10 is also relatively easy, but may not give the most accurate results as I often just end up choosing a random number in the region of important (6-10) or unimportant (which in this case I just left as 1 as I had 5 points already).
\item I liked the first metric (separate scores) most, because I didn't have to do any artificial ranking or make difficult decisions between N things at once. I could. This is interesting though. One factor that will complicate your analysis is that more of the arguments for the first position were compelling than for the second two (though perhaps I'm biased by the format). So direct quantitative comparison between the formats, which doesn't account for the inherent differences in the distribution of argument persuasivenesses, might be misleading.
\item I liked the first metric (separate scores) most, because I didn't have to do any artificial ranking or make difficult decisions between N things at once. I could. This is interesting though. One factor that will complicate your analysis is that more of the arguments for the first position were compelling than for the second two (though perhaps I'm biased by the format). So direct quantitative comparison between the formats, which doesn't account for the inherent differences in the distribution of argument persuasivenesses, might be misleading.
\item Selection was easiest to complete, but I think the rank-ordering will be the most highly informative.
\item Giving a score of 1 to 10 was the most difficult as it was hard to be fair and determine what I thought deserved a certain number. Choosing the 5 best arguments was fairly easy as I didn't really need to rank the statements. Dragging to rank all the choices was somewhat difficult but visually it was simple and easy because I was able to see my choices in order, rather than just attributing them a number 1 through 10.
\item Ranking is the most difficult.  I prefer the format that lets me choose on a scale from 1-10 how strong I think the argument is.  
\item Assigning values from 1-10 was the most difficult voting format.  Simple selection was the easiest, and ranking fell in the middle.  I believe simple selection will produce the most coherent points, but it may be a very small difference between ranking and selection.  Assigning values, while probably being better to analyze statistically, will probably produce the worst bias.
\item I think ranking was the easiest format in this survey
\item The checkbox question was the easiest to complete. The ranking question took the longest - I put greater emphasis on the first 5 ranks and cared less about the remaining 5. In the coring question from 1-10 I felt  that my numbers were fairly arbitrary and I did not need to use the full 1-10 scale (only used scores 1,5,6,7,8,9). 
\item 2/3rds of everything is irrelevant. What matters I suppose is being able to justifiably argue your point in a way which will support the evidence that preceded. Furthermore question the obvious and simply do what you can with what you have in the time you have left (desperado)
\item Question 1's ranking scheme was the easiest as it's easier to visually arrange points. However, Question 2's ranking scheme might produce a better sub-selection of 5 because the focus was on finding only the most relevant points (minimizing cognitive load a bit). 
\item Ranking was more difficult than selection. The selection will result in the most coherent sub-selection of 5 points. The constraint of only choosing the 5 most relevant points necessitates the person to do some sort of ranking. I like the simplicity of selecting the 5 points (the third format presented), but my preferred format of scoring each point from 1 to 10. This allowed me to assign 1 to the points I would never use and also put together a list of the points I would use in my debate. Furthermore, the usable points are now ranked in order, which may be handy when selecting points in a debate.
\item Ranking was more difficult. Particularly when needing to evaluate several equally poor statements, needing to determine an ordering among those was not as easy as simply assigning a score to each statement. I believe that the scoring mechanism will allow you to see definitive separation between stronger and weaker statements as the gap between their position/score can be demonstrably widened.
\item Was confusing that the first two questions went in opposite directions - easy to miss
\item I thought ranking the options (Question 3) was the best approach for comparing arguments. It's easier and faster to compare a given argument with two or one adjacent arguments than it is to judge them in absolute terms.  I think the formats in Questions 2 and 3 are probably equivalent for selecting the best subset of five arguments. I disliked the format in Question 1 because it requires more work to compare different arguments. 
\item the drag-and-drop method facilitates focus on comparing two points when I repeatedly ask myself "is this option better than the one above it?" whereas the "select top 5" makes me compare the option in question to up to 5 others (if I've already selected 5) to decide if it deserves to be selected above one of the others; and the "rate each option" voting mechanism forces me to weigh up the relative strength of each option against all the other options in order to balance my scoring. 
\item Several theorems in game theory and social science [Arrow's, Gibbard–Satterthwaite] state that there's no excellent method of taking preferences from the members of a group and building a set of preferences (or single top choice) that the group would agree with. There might be a loophole around "separate into a top half and a bottom half", but it seems unlikely. Whatever mechanism you decide on will be vulnerable to some particular set of participant votes. That said, it may be possible to find a mechanism that works well for common voting patterns. It's also not clear why collecting LESS data (ordinal or top-5) would ever be more useful than collecting the full scores, unless perhaps the task of scoring 1-10 is more noisy for reasons of cognitive load and greater breaking of IIA. But certainly whatever math is run on the top-5 data could instead be run on the top 5 of the ranking data. Unless the aim of this research is to show that less taxing question formats are less noisy, I don't see the point. Overall, I suspect that I'm about to click through and get told I've been lied to and the real purpose of this research is something else altogether. 
\item I liked the voting format that automatically resorted the choices as they were selected. This made it visually easy to follow the order of 10 items and reorder as needed. I also liked the "choose the best 5" voting format, because I didn't have to select options that I felt were irrelevant, and also because I didn't have to put them in an order and found many choices to be equally as good as another. It was also quick and easy. I disliked the "insert order" voting option because it was cumbersome to go back and re enter numbers and ensure I didn't rank multiple options with the same number. I liked it better when it resorted for me..
\item Drag drop provides the most feedback -- you can immediately see the other items reordered. 
\item I found that question 1 required a fair amount of previous understanding of the US college application process and of acronyms used (such as ACT). I wasn't clear on what a few of the statements meant, not being familiar with the US system myself. I found selecting 5 relevant points the easiest selection, probably because I didn't need to have organized my thoughts as much as the first two that required specific ranking or scoring. I therefore believe that the relevant selection will probably result in the most whereby subsection of 5 points :)
\item I thought tanking was a bit more difficult. Trying to assess the strength of a single sentence set of claims based on a value of 1-10 is arbitrary. I thought it was a fascinating way to ask questions and fruitful to uncover interesting dynamics in the way that questions are answered. 
\end{itemize}


\end{document}
