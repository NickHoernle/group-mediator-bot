%%
%% Copyright 2007, 2008, 2009 Elsevier Ltd
%%
%% This file is part of the 'Elsarticle Bundle'.
%% ---------------------------------------------
%%
%% It may be distributed under the conditions of the LaTeX Project Public
%% License, either version 1.2 of this license or (at your option) any
%% later version.  The latest version of this license is in
%%    http://www.latex-project.org/lppl.txt
%% and version 1.2 or later is part of all distributions of LaTeX
%% version 1999/12/01 or later.
%%
%% The list of all files belonging to the 'Elsarticle Bundle' is
%% given in the file `manifest.txt'.
%%

%% Template article for Elsevier's document class `elsarticle'
%% with harvard style bibliographic references
%% SP 2008/03/01
%%
%%
%%
%% $Id: elsarticle-template-harv.tex 4 2009-10-24 08:22:58Z rishi $
%%
%%
\documentclass[final,authoryear,11pt,times]{elsarticle}

%% Use the option review to obtain double line spacing
%% \documentclass[authoryear,preprint,review,12pt]{elsarticle}

%% Use the options 1p,twocolumn; 3p; 3p,twocolumn; 5p; or 5p,twocolumn
%% for a journal layout:
%% \documentclass[final,authoryear,1p,times]{elsarticle}
%% \documentclass[final,authoryear,1p,times,twocolumn]{elsarticle}
%% \documentclass[final,authoryear,3p,times]{elsarticle}
%% \documentclass[final,authoryear,3p,times,twocolumn]{elsarticle}

%% \documentclass[final,authoryear,5p,times,twocolumn]{elsarticle}

%% if you use PostScript figures in your article
%% use the graphics package for simple commands
%% \usepackage{graphics}
%% or use the graphicx package for more complicated commands
%% \usepackage{graphicx}
%% or use the epsfig package if you prefer to use the old commands
%% \usepackage{epsfig}

%% The amssymb package provides various useful mathematical symbols
\usepackage{amssymb}

\usepackage[margin=1.25in]{geometry}
\parskip=3pt


\usepackage{setspace}
\onehalfspacing

%% The amsthm package provides extended theorem environments
%% \usepackage{amsthm}

%% The lineno packages adds line numbers. Start line numbering with
%% \begin{linenumbers}, end it with \end{linenumbers}. Or switch it on
%% for the whole article with \linenumbers after \end{frontmatter}.
%% \usepackage{lineno}

%% natbib.sty is loaded by default. However, natbib options can be
%% provided with \biboptions{...} command. Following options are
%% valid:

%%   round  -  round parentheses are used (default)
%%   square -  square brackets are used   [option]
%%   curly  -  curly braces are used      {option}
%%   angle  -  angle brackets are used    <option>
%%   semicolon  -  multiple citations separated by semi-colon (default)
%%   colon  - same as semicolon, an earlier confusion
%%   comma  -  separated by comma
%%   authoryear - selects author-year citations (default)
%%   numbers-  selects numerical citations
%%   super  -  numerical citations as superscripts
%%   sort   -  sorts multiple citations according to order in ref. list
%%   sort&compress   -  like sort, but also compresses numerical citations
%%   compress - compresses without sorting
%%   longnamesfirst  -  makes first citation full author list
%%
%% \biboptions{longnamesfirst,comma}

% \biboptions{}

\journal{cs280r - Final Project Report}

\begin{document}

\begin{frontmatter}

\title{CS280r Project Proposal \\ Group Mediation: something}

%% use optional labels to link authors explicitly to addresses:
%% \author[label1,label2]{<author name>}
%% \address[label1]{<address>}
%% \address[label2]{<address>}

\author{Sophie Hilgard \& Nicholas Hoernle}
% \begin{abstract}
% %% Text of abstract

% A paragraph that states (1) the problem addressed, (2) the specific approach taken, (3) the results of that approach, and (optionally) lessons learned. Note: Abstracts are not simply the first paragraph of the introduction.

% \end{abstract}
\end{frontmatter}

\section{Introduction and Motivation}
	\label{sec:introduction-motivation}
	Given a group work setting, a moderator is interested in controlling the interaction of the group while not necessarily participating in the execution of the task at hand \citep{short2015towards}. We propose an on-line meeting environment where an artificial agent is able to control the meeting to optimize for the flow and productivity of the meeting whilst still engaging the various group members. The purpose of this project is to implement a `Slack bot' to test the effect that an influencing agent may have in a human group-meeting environment.

	\citet{matsuyama2015four} highlight the difficulty of maintaining the quality of interaction between a robot agent and a group, due to the difficulties presented in speech analysis of the individuals in the group. Constraining an experiment to an online environment helps to attach identity numbers to individuals but for the purposes of scope, we will further constrain this to a written, online communication forum. Thus to implement a feasible test, we constrain an implementation of the system to slack where easier text analysis can be conducted to facilitate a written group work meeting. The choice of slack is further due to the popularity of the tool in group code and business environments \citep{jeffrey2016scientists}\citep{lebeuf2017software}.

\section{Questions of Interest}
	\label{sec:question_o_i}
	The successful completion of this project will tackle the following questions:
	\begin{itemize}
		\item Given a groupwork environment, can an assistant agent help to facilitate the meeting to increase the interaction of the participants and to optimize for the productivity of the meeting?
		\item Are people amenable to having a meeting managed by an external facilitator?
		\item What are the key considerations when introducing an agent into a human group work planning environment?
	\end{itemize}

\section{Relation to CS280r Coursework}
	\label{sec:coursework}
	\citet{grosz2006dynamics} present the formalism of shared plans but do not elaborate on how humans and/or agents collectively agree upon these plans. \citet{hutchins1995cognition} presents a case study on the formalized communication among the participants about a sailing ship. He clearly shows how the correct communication protocols allowed the team to function efficiently and adapt to a changing environment. \citet{friedkin2016network} introduces the concept of group consensus and how the dependence on logical constraints affect the group decision. This project would rather tackle the problem of neutralizing overly influential (load and outspoken) members of the group and encouraging participation by the quieter/less-spoken group members. The project may further extend the work done by \citet{kamar2009incorporating} who design ways for an agent to decide when it is useful to help other members of the group with assigned tasks. In this case, the agent would have a very incomplete picture of what is being done (as the agent is not expected to comprehend the purposes of the meeting), but it should be in a position to infer enough information to assist in allowing the group to reach a consensus. \\
	In developing the workflow for a given task or meeting, we will likely draw upon the task breakdown and crowdsourcing practices in \citet{hahn2016knowledge} and \citet{chilton2013cascade}. In particular, a meeting mediator will likely source relevant topics that participants believe should be addressed within a meeting (or else perhaps it is provided with a specific task breakdown or meeting agenda). The mediator system would then aggregate the relative importance of these topics, perhaps even having to determine if similar topics are the same, related, and/or nested. Then, as the meeting progresses, the mediator will require some vote-processing scheme. The system will likely be less complicated than those addressed in \citet{benade2016preference} and \citet{procaccia2016voting}, but we'll still be able to draw upon the concepts from these papers.

\section{Division of Work}
	\label{sec:requestedfeedback}
	There is no explicit division of work as we will both tackle the relevant parts of the project as we progress through the project time-line. 
	\begin{enumerate}
		\item{}\textbf{Week 1: April 10-16 -} Work on developing the algorithms and ideal interface for the core contribution: a system that, given a task or a set of topics, can filter messages from participants to feature those that are most likely to lead to a resolution based on the feedback of meeting participants. If we time at the end, we would also work to develop a system that can generate the meeting agenda itself given topics elicited from participants. Begin looking into Slack API	
		\item{}\textbf{Week 2: April 17-23 -} Implement Slack interface 
		\item{}\textbf{Week 3: April 24-30 -} Testing and refinement of interface, project presentations
		\item{}\textbf{Week 4: May 1-7 -} Development of project-related experiment and testing on friends/classmates to get possible results concerning effectiveness at task completion. Refinement of project paper, which we will have been adding to throughout the timeline.	
	\end{enumerate}

\section{Requested Feedback}
	\label{sec:requestedfeedback}
	\begin{itemize}
		\item{}What interface would you be willing to interact with in a setting like this? 
		\item{}Should we attempt to have the bot crowdsource the meeting agenda as well as manage feedback during the meeting? 
		\item{}How would you develop the priority queue for messages - in particular, is there a concern regarding possible lack of cohesiveness if messages are delayed from when they were originally sent? And if so, does this just require reasonable adaptation by the users or is it too unnatural to accept?
	\end{itemize}

\section{References}

\bibliographystyle{elsarticle-num-names}
\bibliography{proposal}

\end{document}


\end{document}
