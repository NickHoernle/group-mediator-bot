%%
%% Copyright 2007, 2008, 2009 Elsevier Ltd
%%
%% This file is part of the 'Elsarticle Bundle'.
%% ---------------------------------------------
%%
%% It may be distributed under the conditions of the LaTeX Project Public
%% License, either version 1.2 of this license or (at your option) any
%% later version.  The latest version of this license is in
%%    http://www.latex-project.org/lppl.txt
%% and version 1.2 or later is part of all distributions of LaTeX
%% version 1999/12/01 or later.
%%
%% The list of all files belonging to the 'Elsarticle Bundle' is
%% given in the file `manifest.txt'.
%%

%% Template article for Elsevier's document class `elsarticle'
%% with harvard style bibliographic references
%% SP 2008/03/01
%%
%%
%%
%% $Id: elsarticle-template-harv.tex 4 2009-10-24 08:22:58Z rishi $
%%
%%
\documentclass[final,authoryear,11pt,times]{elsarticle}


%% Use the option review to obtain double line spacing
%% \documentclass[authoryear,preprint,review,12pt]{elsarticle}

%% Use the options 1p,twocolumn; 3p; 3p,twocolumn; 5p; or 5p,twocolumn
%% for a journal layout:
%% \documentclass[final,authoryear,1p,times]{elsarticle}
%% \documentclass[final,authoryear,1p,times,twocolumn]{elsarticle}
%% \documentclass[final,authoryear,3p,times]{elsarticle}
%% \documentclass[final,authoryear,3p,times,twocolumn]{elsarticle}

%% \documentclass[final,authoryear,5p,times,twocolumn]{elsarticle}

%% if you use PostScript figures in your article
%% use the graphics package for simple commands
%% \usepackage{graphics}
%% or use the graphicx package for more complicated commands
%% \usepackage{graphicx}
%% or use the epsfig package if you prefer to use the old commands
%% \usepackage{epsfig}

%% The amssymb package provides various useful mathematical symbols
\usepackage{amssymb}

\usepackage[margin=1.25in]{geometry}
\parskip=3pt


\usepackage{setspace}
\onehalfspacing

\usepackage{enumitem}

%% The amsthm package provides extended theorem environments
%% \usepackage{amsthm}

%% The lineno packages adds line numbers. Start line numbering with
%% \begin{linenumbers}, end it with \end{linenumbers}. Or switch it on
%% for the whole article with \linenumbers after \end{frontmatter}.
%% \usepackage{lineno}

%% natbib.sty is loaded by default. However, natbib options can be
%% provided with \biboptions{...} command. Following options are
%% valid:

%%   round  -  round parentheses are used (default)
%%   square -  square brackets are used   [option]
%%   curly  -  curly braces are used      {option}
%%   angle  -  angle brackets are used    <option>
%%   semicolon  -  multiple citations separated by semi-colon (default)
%%   colon  - same as semicolon, an earlier confusion
%%   comma  -  separated by comma
%%   authoryear - selects author-year citations (default)
%%   numbers-  selects numerical citations
%%   super  -  numerical citations as superscripts
%%   sort   -  sorts multiple citations according to order in ref. list
%%   sort&compress   -  like sort, but also compresses numerical citations
%%   compress - compresses without sorting
%%   longnamesfirst  -  makes first citation full author list
%%
%% \biboptions{longnamesfirst,comma}

% \biboptions{}

\journal{cs280r - Project Update}

\begin{document}

\begin{frontmatter}

\title{CS280r Project Update \\ Group Meeting Facilitation: Implementation of an Artificial Mediator}

%% use optional labels to link authors explicitly to addresses:
%% \author[label1,label2]{<author name>}
%% \address[label1]{<address>}
%% \address[label2]{<address>}

\author{Sophie Hilgard \& Nicholas Hoernle}
% \begin{abstract}
% %% Text of abstract

% A paragraph that states (1) the problem addressed, (2) the specific approach taken, (3) the results of that approach, and (optionally) lessons learned. Note: Abstracts are not simply the first paragraph of the introduction.

% \end{abstract}
\end{frontmatter}

\section{Update}
We have decided to focus on the agenda setting goal from the project proposal. As the scope of this task is still fairly broad, after speaking to Barbara, we have decided to study a content generation platform that could in theory be applied to the meeting agenda setting. The power of crowd-sourcing is such that we are aiming to source opinions from the entire team due to the importance of including items in the agenda that reflect the needs of the individuals in the team \citep{schwartz2015design}. We have seen that other content-generation systems, such as Wikipedia, often end up biased to the desires of a single coordinator, which we see as a parallel to a biased/suboptimal voting system in a domain without an absolute ground truth. With this in mind, we will consider the use of alternative voting-based methods for selecting maximally relevant topics within a time/space budget.

Specifically, we will be building on work from \citet{hahn2016knowledge} and \citep{caragiannis2017subset} to select the maximally relevant subset to be studied further (or included in a meeting agenda) from a list of possible options. As the meeting setting is highly qualitative and requires too much context to run a successful experiment, we will experiment instead on generating the most relevant subset of points from a given opinion news article. Readers will be asked to either rank or identify in a binary manner those points which they think could build the most convincing argument. We will then use two different voting rules to aggregate the opinions of the subjects to create the optimal subsets. The one voting rule will be based on the work by \citet{caragiannis2017subset} and one will be the naive approach of selecting whichever topics receive the highest number of votes in an approval voting setting (with ties broken at random). We can also compare this against a baseline of an individual creating a topic summary of the entire document. To test which output is of highest quality, we can ask a separate group which of the topic lists they would prefer to use to construct an argument (or perhaps alternatively we can reconstruct the opinion piece with only the relevant topics left in and ask a new set of people whether they find the argument convincing and report the percentage for each group).

\bibliographystyle{elsarticle-num-names}
\bibliography{update}

\end{document}


\end{document}
